\section{The setting}

\underline{Data:}

observations: $\big\{ \vec{x}^{(\alpha)} \big\}, \alpha = 1, \ldots, p; \quad \vec{x} \in \R^N$

$$
\vec x = \rmat{
x_1\\
x_2\\
\vdots\;\,\\
x_N
}
$$

Our entire dataset:
\[
\vec X = 
\left(
\begin{array}{cccc}
\Big| & \Big| & &\Big| \\[3mm]
\vec x^{(1)} & \vec x^{(2)} & \cdots &\vec x^{(p)}\\[2mm]
\Big| & \Big| & &\Big|
\end{array}
\right) \in \R^{N \times p}
\]

\section{Dimensionality Reduction}
\vspace{0.5cm}

We want to reduce the number of elements in $\vec x \in \R^N$
while retaining as most of the intrinsic information content.

\question{For what purpose?}

\question{What is the difference between dimensionality reduction and compression, if any?}

\newpage

\underline{Simple truncation:}

Let\\[0.2cm]
$\vec x \in \R^N$,\\
$\E[\vec x] = \vec 0$ (i.e. for each variable $x_i$ its mean $m_i = \sum_{\alpha=1}^{p} x_i^{(\alpha)} = 0$) and $1 < M\, < N$.\\[0.2cm]
We simply transmit the first $M$ elements of $\vec x$. The recipient reconstructs all $N$ elements by adding zero entries for all missing elements (i.e. \textit{zero-padding}).\\[0.2cm]
How much error will they accumulate in this case?\\[0.2cm]
Let $\widetilde{\vec{x}}$ be the reconstructed observation. The first $M$ elements were transmitted perfectly, zero padding is used to extend the vector to its original size of $N$ elements:
\begin{align*}
MSE  &=  \frac{1}{p} \sum\limits_{\alpha = 1}^p ( \vec{x}^{(\alpha)} - \widetilde{\vec{x}}^{(\alpha)} )^2\\
     &=  \frac{1}{p} \sum\limits_{\alpha = 1}^p \left(\underbrace{\sum\limits_{j = 1}^M ( x_j^{(\alpha)} - \widetilde{x_j}^{(\alpha)} )^2}_{\substack{=0 \\\text{ (perfect transmission)}}} + \sum\limits_{j = M+1}^N ( x_j^{(\alpha)} - \underbrace{\widetilde{x_j}^{(\alpha)}}_{\substack{=0\\ \text{padded}}}\;)^2 \right)\\
     &=  \frac{1}{p} \sum\limits_{\alpha = 1}^p \sum\limits_{j = M+1}^N ( x_j^{(\alpha)} )^2 \\
     &=  \sum\limits_{j = M+1}^N \frac{1}{p} \sum\limits_{\alpha = 1}^p  ( x_j^{(\alpha)} )^2 \\
     &=  \sum\limits_{j = M+1}^N \sigma_j^2 \\
\end{align*}
They will end up with an MSE equal to $\sum_{j=M+1}^{N} \sigma_j^2$, where $\sigma_j^2$ is the variance of the $x_j$.\\

\underline{Objective:} Rotate/Transform the data s.t. truncating the transformed vector $\vec v \in \R^M$ is optimum in the sense of minimal MSE.

\newpage
\section{PCA}
\begin{enumerate}
\item Center the data, $\E[\vec x] = \vec m  = \frac{1}{p} \sum_{\alpha=1}^{p} \vec x^{(\alpha)}\eqexcl \vec 0$.
\item Let $\vec X'$ be the $N \times p$ matrix of the centered data.
\item Measure the variance of each component in $\vec x_{centered}$. Not enough, the variables in $\vec x$ could be correlated.
\item Measure covariances $C_{ij}$.
\item Construct covariance matrix $\vec C$.\\
$$ 
\vec C = \text{Cov}(\vec X') = \mathbf{\Sigma} = \E[\vec X' \vec X'^\top] \in \R^{N \times N}
$$
\item \textbf{eigenvalue decomposition}:\\
$$
\vec C \, \vec e_a \; = \; \lambda_a \vec e_a  \quad\text{(the eigenvalue problem of PCA)}
$$
$$
\lambda_a: \text{the eigenvalue, the variance along principle component } a.
$$
$$
\vec e_a: \text{the normalized eigenvector, the direction of PC } a \text{ in } R^N
$$

\question{How do we perform an eigenvalue decomposition?}
((1) Get eigenvalues: $\det(\vec C-\lambda \vec I) = 0$, 
(2) Find eigenvector $\vec e_a$ associated with each $\lambda_a$ by solving the linear system\\
$ (\vec C - \lambda_a \vec I )\, \vec e_a = \vec 0$
(see {\emph{math\_primer.pdf} on ISIS}  for details)
\item Order eigenvalues in descending order. (Highest variance first). The ordered eigenvectors are the \emph{principle components} of the dataset $\vec X$.

\question{What is a scree plot?}

\item Rotate $\vec x_{centered}$ using the first $M$ PCs.
\end{enumerate}

%\newpage

\underline{How much better is this than simple trunctation?}\\[0.3cm]
The transformation onto the PCs is linear:
\begin{equation*}
	\vec{x} = \underbrace{ a_1 }_{ \vec{e}_1^\top \vec{x} } \vec{e}_1
		+ \underbrace{ a_2 }_{ \vec{e}_2^\top \vec{x} } \vec{e}_2
		+ \ldots
		+ \underbrace{ a_N }_{ \vec{e}_N^\top \vec{x} } \vec{e}_N
\end{equation*}

Therefore, we perfectly reconstruct observations by projecting them from PC space (i.e. feature space) back to the input space. 
However, if we only use the first $M$ PCs for reconstructing the observations, we will accumulate error.

$$
\widetilde{\vec{x}} = a_1 \vec{e}_1 + a_2 \vec{e}_2 + \ldots
		+ a_M \vec{e}_M
$$
		
We measure MSE between original centered observations and their corresponding reconstructions.
$$
E = \frac{1}{p} \sum_{\alpha = 1}^{p} (\vec x_{centered} - \widetilde{\vec{x}})^2 = \frac{1}{p} \sum_{\alpha = 1}^{p} \sum_{j = M+1}^{N} (a_j^{(\alpha)})^2
$$
The MSE is equal to the sum of variances of the final M components of the \emph{transformed} observations. Since we've ordered the PCs w.r.t to variance in descending order the variance of the last $M-N$ components of the transformed data is smallest.
The transformation is therefore optimal in the sense of minimal MSE.


\subsection{Applying PCA}
What can we do with PCA?
\begin{enumerate}
\item Dimensionality reduction
\item Visualization (e.g. plot the projections onto the first 2 or 3 PCs.)
\item Whitening (i.e. decorrelate the data) - exploit that eigenvectors form an orthonormal basis.
$$
\vec v^{(\alpha)} = \vec x_{whitened}^{(\alpha)} = \vec{\Lambda}^{-\frac{1}{2}}\vec{M}^\top\vec{x}^{(\alpha)}
\quad
\forall \alpha
$$

\question{What is the covariance of $\vec x_{whitened}$?}

\end{enumerate}

\newpage
\subsection{A note on implementation:}

\question{Is an off-the-shelf implementation such as\\ sklearn.decomposition.PCA for python really necessary?}

PCA boils down to solving the eigenvalue problem.
We bascially need a routine with which to solve the eigenvalue problem. For python, the obvious choice is to use a routine like \textit{numpy.linalg.eig(A)}. \emph{Whatch out for how to intepret the output.}

\question{Are there alternatives to eigenvalue decomposition?}

Yes, namely,

\underline{Singular Value Decomposition (SVD)}:\\
Let $\vec X' \in \R^{N \times p}$ be the centered dataset (centering is a must):
$$
\vec X'_{N \times p} = \vec U_{N \times N} \, \vec S_{N \times p} \, \vec V^\top_{{p \times p}}
$$
where,
$$
\vec U^\top \vec U = \vec I_{N \times N}
$$ and 
$$
\vec V^\top \vec V = \vec I_{p \times p}
$$ (i,e. $\vec U$, $\vec V$ are orthogonal.).
\begin{itemize}
\item The columns of $\vec U$ make up the eigenvectors of $\vec X'\vec X'^\top$.
\item The columns of $\vec V$ (or rows of $\vec V^\top$) make up the eigenvectors of $\vec X'^\top\vec X'$.
\item $\vec S$ is a diagonal matrix. The singular values in $\vec S$ (the diagonal entries) are the \textbf{square roots} of the  eigenvalues of the eigenvectors of $\vec X'\vec X'^\top$ or $\vec X'^\top\vec X'$.

\end{itemize}
Recall:

$$
\vec C = \E[\vec X \vec X^\top] = \frac{1}{p} \vec X \vec X^\top
$$

\underline{When to use SVD vs. Eig? \textbf{Not dogma}}

\begin{itemize}

\item Numerical instabilities in eigendecomposition of very large covariance matrix $\rightarrow$ SVD
\item SVD not applicable to Kernel-PCA $\rightarrow$ Eig
\item Computational efficiency $\rightarrow$ SVD (depends...?)
\end{itemize}
