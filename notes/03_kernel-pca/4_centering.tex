\subsection{Centering the kernel matrix}

\begin{frame}{\subsecname}

PCA operates on centered data. This is common for all forms  of PCA, 
whether it is linear PCA, online PCA or Kernel PCA.\\
In Kernel PCA, the $\vec X$ is transformed into a higher dimensional space by applying the kernel trick.
Applying it to all $p$ training samples yields the kernel matrix, which is used in solving the transformed eigenvalue problem.

All of the above assumed that we have successfully centered the kernel matrix. We now look at how this is done.

\end{frame}


\begin{frame}{\subsecname}


\question{Do I need to center $\vec X$ before computing $ K_{\alpha \beta}$?}

\pause

- No, Centering $\vec X$ before computing $K_{\alpha \beta}$ does not guarantee the kernel matrix to be centered. You only end up with $\vec{\widetilde K}$. The same applies if we had the $\phi$'s.

It is simply irrelevant whether the original data $\vec X$ is centered or not. 
Therefore, we need to center the kernel matrix before solving the transformed eigenvalue problem.\\

\end{frame}

\begin{frame}{\subsecname}

\notesonly{
We use $\vec{\widetilde K}$ to denote the \emph{un-normalized} kernel matrix and $\vec K$ 
to denote the kernel matrix \emph{after} it has been centered.\\
}

%\underline{Centering $\vec{\widetilde K}$:}

The centering is built on the kernel trick\notesonly{ as in Eq.\ref{eq:trick}}:
\begin{equation}
      k(\vec{x}, \vec{x}') = K_{\vec x,\vec x'} = \vec{\phi}_{(\vec{x})}^\top 
		\vec{\phi}_{(\vec{x}')},
\end{equation}

except that we compute the inner product using \emph{centered} $\phi_{(\vec x)}$:

\begin{equation}
	\underbrace{ \vec{\phi}_{\big( \vec{x}^{(\alpha)} \big)} }_{
		\substack{ 	\text{centered} \\
				\text{feature vectors}} }
				\hspace{-3mm}
	= \widetilde{\vec{\phi}}_{\big( \vec{x}^{(\alpha)} \big)}
				\hspace{-1mm}
		- \frac{1}{p} \sum\limits_{\gamma = 1}^p 
				\hspace{-2mm}
		\underbrace{ \widetilde{\vec{\phi}}_{\big( \vec{x}^{(\gamma)} 
				\big)} }_{
			\substack{ 	\text{uncentered} \\
					\text{feature vectors}} }
					\qquad \forall \alpha
\end{equation}

\notesonly{
Plugging \emph{centered} $\vec{\phi}_{(\vec{x})}$ into the kernel trick equation\slidesonly{ above} reveals how to obtain centered elements $K_{\alpha\beta}$ for }forming a \emph{centered} \notesonly{Kernel matrix }$\vec K$ from an \emph{unnormalized} \notesonly{Kernel matrix }$\widetilde{\vec K}$.

\begin{equation}
	K_{\alpha \beta} = \underbrace{\widetilde{K}_{\alpha \beta}}_{ _{= \, k \left ( \vec{x}^{(\alpha)},
			\vec{x}^{(\beta)} \right )}}
		- \;\underbrace{\frac{1}{p} \sum\limits_{\delta = 1}^p 
		\widetilde{K}_{\alpha \delta}}_\text{\scriptsize{row avg.}} 
		\; - \; \underbrace{\frac{1}{p} 
		\sum\limits_{\gamma = 1}^p 
		\widetilde{K}_{\gamma \beta}}_\text{\scriptsize{col. avg.}}
		\;+ \;\underbrace{\frac{1}{p^2} 
		\sum\limits_{\gamma, \delta = 1}^p 
		\widetilde{K}_{\gamma \delta}}_\text{\scriptsize{matrix avg.}}
\end{equation}%


\end{frame}

