\subsection{Whitening in the context of ICA}

\begin{frame}

\begin{align}
\label{eq:expxina}
\vec \Sigma_x = \mathrm{Cov}(\vec x) &=  \E \lbrack \, \vec x \, \vec x^\top \rbrack\\
&=  \E \lbrack \, \vec A\,\vec s \, \left( \vec A\,\vec s \right)^\top \rbrack\\
&=  \E \lbrack \, \vec A\; \underbrace{\;\vec s \, \vec s^\top}_{= \vec I_N} \vec A^\top \rbrack\\
\label{eq:sigmax}
&=  \vec A\, \vec A^\top
\end{align}

\end{frame}

\begin{frame}

\notesonly{
If we were to apply \emph{Singular Value Decomposition} on the symmetric matrix $\Sigma_x$:
}

\begin{equation}
\mathrm{SVD}(\vec \Sigma_x) = \vec U\, \vec D \, \vec U^\top
\end{equation}

where\\
\notesonly{
$\vec U$ is an orthogonal matrix,} $\vec U^\top\vec U = \vec I_N$ and
\notesonly{$\vec D$ is a diagonal matrix with rank $N$, }$\vec D^\top = \vec D$.\\

We define the two following transformation $\vec Q$ and $\widetilde{\vec A}$:
\begin{equation}
\label{eq:defq}
\vec Q := \vec D^{-\frac{1}{2}} \vec U^\top
\end{equation}
\begin{equation}
\label{eq:defatilde}
\widetilde{\vec A} := \vec Q \, \vec A 
\end{equation}

Which yields a new ICA problem:
\begin{equation}
\vec u = \vec Q\, \vec x = \vec Q\,\vec A \, \vec s = \widetilde{\vec A} \, \vec s
\end{equation}

\question{What is so special about the new mixing matrix $\widetilde{\vec A}$?} \\

\notesonly{-$\widetilde{\vec A}$ is orthogonal (i.e. $\widetilde{\vec A}\, \widetilde{\vec A}^\top = \widetilde{\vec A}^\top \widetilde{\vec A} = \widetilde{\vec A}^{-1} \widetilde{\vec A} = \vec I_N$).
 We will see the implications of this in how it reduces the number of free parameters for solving the new ICA problem.
 }
\end{frame}

\begin{frame}

\slidesonly{
$$
\vec Q := \vec D^{-\frac{1}{2}} \vec U^\top
$$
$$
\vec \Sigma_x = \E \lbrack\, \vec x \, \vec x^\top \rbrack = \vec U \, \vec D \, \vec U^{\top}
$$
}

\begin{align}
\E \lbrack\, \vec u \, \vec u^\top \rbrack
&\mystackrel{\vec u = \vec Q\, \vec x}{\vec u = \vec Q\, \vec x}
 \E \lbrack\, \vec Q \, \vec x  \left( \vec Q \, \vec x\right)^\top \rbrack\\
&\mystackrel{\vec u = \vec Q\, \vec x}{}\E \lbrack\, \vec Q \, \vec x \; \vec x^\top \vec Q^\top \rbrack\\
&\mystackrel{\vec u = \vec Q\, \vec x}{} \E \lbrack\, \vec Q \, \vec \Sigma_{X} \, \vec Q^\top \rbrack\\
&\mystackrel{\vec u = \vec Q\, \vec x}{} \vec Q \, \vec \Sigma_{x} \, \vec Q^\top\\
&\mystackrel{\vec u = \vec Q\, \vec x}{}
\left( \vec D^{-\frac{1}{2}} \vec U^\top \right)
\left( \vec U \, \vec D \, \vec U^\top \right)
\left( \vec D^{-\frac{1}{2}} \vec U^\top \right)
\end{align}

\end{frame}
\begin{frame}

\slidesonly{
\begin{align}
\E \lbrack\, \vec u \, \vec u^\top \rbrack
&\mystackrel{\vec u = \vec Q\, \vec x}{}
\left( \vec D^{-\frac{1}{2}} \vec U^\top \right)
\left( \vec U \, \vec D \, \vec U^\top \right)
\left( \vec D^{-\frac{1}{2}} \vec U^\top \right)
\end{align}
}

From $\vec U^\top\vec U = \vec I_N$ and $\vec D^\top = \vec D$ follows:

\begin{align}
\E \lbrack\, \vec u \, \vec u^\top \rbrack
&=
\vec D^{-\frac{1}{2}}
\underbrace{\left( \vec U^\top  \vec U \right)}_{= \vec I_N}
\, \vec D \, 
\underbrace{\left( \vec U^\top \,  \vec U \right)}_{= \vec I_N}
\,
\left(\vec D^{-\frac{1}{2}} \right)^\top \\
&= \vec D^\top
\, 
\underbrace{\vec D^{-\frac{1}{2}}  \, 
\left(\vec D^{-\frac{1}{2}} \right)^\top}_{=\vec D^{-1}} = \vec I_N
\end{align}


Recall 
\notesonly{from \eqref{eq:sigmax}:}
$$
\vec \Sigma_x =  \E \lbrack \, \vec x \, \vec x^\top \rbrack
=  \vec A\, \vec A^\top
$$
Therefore:
$$
\E \lbrack \, \vec u \, \vec u^\top \rbrack
= \widetilde {\vec A}\, \widetilde {\vec A}^\top = \vec I_N
$$

\notesonly{
The new mixing matrix}
$\widetilde{\vec A}$ is orthonormal.\\
The space of solutions is restricted to $\vec W$ that are orthogonal. 
The can speed up convergence of ICA. 
The number of free parameters 
\notesonly{
for solving the new ICA problem 
$\vec u = \widetilde{\vec A}\, \vec s$ 
is reduced from $N^2$ to $N(N-1)/2$
}
\slidesonly{
$N^2 \rightarrow N(N-1)/2$
}
\end{frame}

\begin{frame}
\slidesonly{
\frametitle{What just happened?}

\begin{itemize}
\setlength\itemsep{1em}
\item We've transformed $\vec x$ to get $\vec u$:
$ \qquad \qquad \qquad
\vec u := \vec D^{-\frac{1}{2}} \vec U^\top \vec x
$

\item Where did $\vec D$ and $\vec U$ come from?
$ \qquad \qquad
\mathrm{SVD}(\vec \Sigma_x) = \vec U\, \vec D \, \vec U^\top
$

\item What do $\vec D$ and $\vec U$ represent? - the eigenval. and eigenvect. of $\vec \Sigma_x$

\item What's so special about $\vec u$?
$ \qquad \qquad \qquad \qquad
\vec \Sigma_u = \vec I_N
$

\item ...So?

- new ICA problem: $\vec u := \widetilde{\vec A}\, \vec s$\\
- $\widetilde{\vec A}$ is orthogonal\\
- unmixing the new problem involves only ``half'' the number of weights.

\question{What is the transformation $\vec D^{-\frac{1}{2}} \vec U^\top$ called?}

\end{itemize}
}

\end{frame}


\section{Summary so far:}

\slidesonly{
\begin{frame}
\begin{enumerate}
\item Initial ICA Problem: $\vec x = \vec A\, \vec s$
\item New ICA Problem: $\vec u := \widetilde{\vec A}\, \vec s$,\\

where $\vec u = \vec D^{-\frac{1}{2}} \vec U^\top \vec x$ and $\vec \Sigma_u = \vec I_N$.
\item $\vec u$ is the \emph{whitened} version of $\vec x$.
\item $\vec D$ and $\vec U$ can be obtained via PCA on $\vec x$.
\item Applying ICA on whitened data reduces the number of free parameters.
\item PCA simplifies the ICA problem.
\item ICA on whitened data is about ``rotating'' it back.
\end{enumerate}

\end{frame}
}

