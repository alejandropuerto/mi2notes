
\section{The ICA problem}

Let $\vec s = (s_1, s_2,...,s_N)^\top$ denote the concatenation of independent sources 
and $\vec x \in \R^N$ describe our observations. $\vec x$ relates to $\vec s$ through a 
\emph{linear transformation} $\vec A$:

\begin{equation}
\label{eq:ica}
\vec x = \vec A \, \vec s.
\end{equation}

We refer to $\vec A$ as the \emph{mixing matrix} and Eq.\ref{eq:ica} as the \emph{ICA problem}, 
which is recovering $\vec s$ from only observing $\vec x$.

\underline{Example scenario:}

Two speakers are placed in a room and emit signals $s_1$ and $s_2$. 
The speakers operate indepdendent of one another.  
Two microphones are placed in the room and start recording. 
The first microphone is placed slightly closer to speaker 2, while 
the second microphone is placed slightly closer to speaker 1.
$x_1$ and $x_2$ denote the recordings of the first and second microphone respectively. 
When we listen to the recordings we expect to hear a mix of $s_1$ and $s_2$. 
Since microphone 1 was placed closer to speaker 2, when we only listen to $x_1$ we hear more of $s_2$ than $s_1$. 
The opposite can be said when we listen only to $x_2$.

Acoustic systems are linear. This means that $x_1$ is a superposition of \emph{both} sources $s_1$ and $s_2$. 
We will assume here that the contribution of a source $s_i$ 
to an observation $x_j$ is inversely proportional to the distance between the source and the microphone. 
The distance-contribution relationship is \emph{linear}. We don't need this to be any more realistic. 

If we had a measurement of the distance between each microphone and each speaker, 
we would tell exactly what the contribution of each of $s_1$ and $s_2$ is to each recorded observation. 
If we know the exact contribution of a source to an observation, we can look at both observations and recover each source in full.

This is what ICA tries to solve, except that it does not have any knowledge about the spatial setting. It is blind.
