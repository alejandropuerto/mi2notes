\section{Density Transformation}

\underline{Outline:}

Before we tackle ICA itself, we first look at the more basic principle of \emph{density transformation} and 
the \emph{convservation of probability}.\\
We start with more specific cases of applying density transformations, 
namely \emph{pseudo random number generators} and what the inverse of \emph{cumulative distribution functions (cdf)} can be used for.\\
Finally we discuss how to generalize this in order to transform one probability density function (pdf) into another.

\subsection{PRNG:}

How can we sample from the uniform distribution $\in \lbrack0, 1)$?

\begin{itemize}
\item Create a sequence, preferrably with a wrong period.
\item minimal pattern and sub-subsequences
\item determinism has an advantage:
\begin{enumerate}
	\item \emph{reproducible} sequneces
	\item efficiency; the starting element or ``seed'' and length of the sequence is suffcient is representative of the entire sequnece.
\end{enumerate}
\end{itemize}

\underline{Linear congruential generator (LCG):}

Start with a seed $y_0 \in \overbrace{\left\{0,\ldots,m-1\right\}}^{=:\;\mathcal{M}}$ with $m \in \N$ ($m$ controls the granularity). 
The next sample $y_t$ is computed as:

\begin{equation}
y_t = \left( \, a \; y_{t-1} \; + \; b \, \right) \, \text{mod} \; m,
\end{equation}
where\\[-0.7cm]
\begin{align*}
a \in \mathcal{M}&\; \text{is the multiplier,} \\
b \in \mathcal{M}&\; \text{is the increment.}
\end{align*}

Then $u_i = \frac{y_i}{m} \approx \,\mathcal{U} \in \lbrack0, 1)$.

Although finicky and requiring careful parameterization, LCG gives us something for drawing from a uniform distribution. 
Next, we look at how to draw samples of a random variable $X$ with a desired pdf $p_X(x)$. 
using uniformly sampled values $\tilde z \in [0,1]$.

